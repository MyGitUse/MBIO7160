\documentclass[]{article}
\usepackage{lmodern}
\usepackage{amssymb,amsmath}
\usepackage{ifxetex,ifluatex}
\usepackage{fixltx2e} % provides \textsubscript
\ifnum 0\ifxetex 1\fi\ifluatex 1\fi=0 % if pdftex
  \usepackage[T1]{fontenc}
  \usepackage[utf8]{inputenc}
\else % if luatex or xelatex
  \ifxetex
    \usepackage{mathspec}
  \else
    \usepackage{fontspec}
  \fi
  \defaultfontfeatures{Ligatures=TeX,Scale=MatchLowercase}
\fi
% use upquote if available, for straight quotes in verbatim environments
\IfFileExists{upquote.sty}{\usepackage{upquote}}{}
% use microtype if available
\IfFileExists{microtype.sty}{%
\usepackage{microtype}
\UseMicrotypeSet[protrusion]{basicmath} % disable protrusion for tt fonts
}{}
\usepackage[margin=1in]{geometry}
\usepackage{hyperref}
\hypersetup{unicode=true,
            pdftitle={MBIO7160 Project Proposal},
            pdfborder={0 0 0},
            breaklinks=true}
\urlstyle{same}  % don't use monospace font for urls
\usepackage{graphicx,grffile}
\makeatletter
\def\maxwidth{\ifdim\Gin@nat@width>\linewidth\linewidth\else\Gin@nat@width\fi}
\def\maxheight{\ifdim\Gin@nat@height>\textheight\textheight\else\Gin@nat@height\fi}
\makeatother
% Scale images if necessary, so that they will not overflow the page
% margins by default, and it is still possible to overwrite the defaults
% using explicit options in \includegraphics[width, height, ...]{}
\setkeys{Gin}{width=\maxwidth,height=\maxheight,keepaspectratio}
\IfFileExists{parskip.sty}{%
\usepackage{parskip}
}{% else
\setlength{\parindent}{0pt}
\setlength{\parskip}{6pt plus 2pt minus 1pt}
}
\setlength{\emergencystretch}{3em}  % prevent overfull lines
\providecommand{\tightlist}{%
  \setlength{\itemsep}{0pt}\setlength{\parskip}{0pt}}
\setcounter{secnumdepth}{0}
% Redefines (sub)paragraphs to behave more like sections
\ifx\paragraph\undefined\else
\let\oldparagraph\paragraph
\renewcommand{\paragraph}[1]{\oldparagraph{#1}\mbox{}}
\fi
\ifx\subparagraph\undefined\else
\let\oldsubparagraph\subparagraph
\renewcommand{\subparagraph}[1]{\oldsubparagraph{#1}\mbox{}}
\fi

%%% Use protect on footnotes to avoid problems with footnotes in titles
\let\rmarkdownfootnote\footnote%
\def\footnote{\protect\rmarkdownfootnote}

%%% Change title format to be more compact
\usepackage{titling}

% Create subtitle command for use in maketitle
\providecommand{\subtitle}[1]{
  \posttitle{
    \begin{center}\large#1\end{center}
    }
}

\setlength{\droptitle}{-2em}

  \title{MBIO7160 Project Proposal}
    \pretitle{\vspace{\droptitle}\centering\huge}
  \posttitle{\par}
    \author{}
    \preauthor{}\postauthor{}
    \date{}
    \predate{}\postdate{}
  

\begin{document}
\maketitle

\begin{itemize}
\tightlist
\item
  \textbf{Question:} Principal component analysis (PCA) is commonly used
  in microbiome \& mycobiome analysis with many papers not providing a
  clear interpretation of the results. How does PCA reduce data into
  dimensions and what is the interpretation?

  \begin{itemize}
  \tightlist
  \item
    Textbook chapter of relevance: CH7 Multivariate Analysis
  \item
    Applications: Comparing beta-diversity between two different sets of
    data (has many problems in itself) or variables within a single
    study(Healthy vs Diseased)

    \begin{itemize}
    \tightlist
    \item
      Other applications, limitations, assumptions, alternatives: TBD
      (to be done)
    \end{itemize}
  \end{itemize}
\item
  Proposed dataset(s)

  \begin{itemize}
  \tightlist
  \item
    \href{https://www.ncbi.nlm.nih.gov/bioproject/?term=PRJNA419104}{Gut
    fungal dysbiosis correlates with reduced efficacy of fecal
    microbiota transplantation in Clostridium difficile infection}

    \begin{itemize}
    \tightlist
    \item
      Brief background: Study was performed to view differences between
      the gut microbiome (bacterial) vs gut mycobiome (fungal) in
      patients before and after fecal transplants to see the affect on
      recurrent C. difficile infections. This BioProject should contain
      sequencing data for controls + cases.
    \end{itemize}
  \item
    \href{https://www.ncbi.nlm.nih.gov/bioproject/PRJNA356769/}{The gut
    mycobiome of the Human Microbiome Project healthy cohort}

    \begin{itemize}
    \tightlist
    \item
      Breif background: This study wanted to see if there is a consensus
      ``normobiota'' of the fungal mycobiome by sequencing a healthy
      cohort and characterizing their mycobiota.
    \end{itemize}
  \end{itemize}
\item
  Project Idea (3-4 sentences)

  \begin{itemize}
  \tightlist
  \item
    I will take the data from
  \end{itemize}
\end{itemize}

installr::installr() Sys.setenv(PATH = paste(Sys.getenv(``PATH''),
``C:/Users/nmok/AppData/Local/Programs/MiKTeX 2.9/miktex/bin/x64'',
sep=.Platform\$path.sep))


\end{document}
